\chapter{Exercise 23: Meet Duff's Device}

This exercise is a brain teaser where I introduce you to one of the
most famous hacks in C called "Duff's Device", named after Tom Duff
the "inventor".  This little slice of awesome (evil?) has nearly everything
you've been learning wrapped in one tiny little package.  Figuring out
how it works is also a good fun puzzle.

\begin{aside}{This Is Only An Exercise}
Part of the fun of C is that you can come up with crazy hacks like this,
but this is also what makes C annoying to use.  It's good to learn about
these tricks because it gives you a deeper understanding of
the language and your computer.  But, you should never use this.  Always
strive for easy to read code.
\end{aside}

Duff's device was "discovered" (created?) by Tom Duff and is a trick
with the C compiler that actually shouldn't work.  I won't tell you what
it does yet since this is meant to be a puzzle for you to ponder and
try to solve.  You are to get this code running and then try to figure
out what it does, and \emph{why} it does it this way.

\begin{code}{ex23.c}
<< d['code/ex23.c|pyg|l'] >>
\end{code}

In this code I have three versions of a copy function:

\begin{description}
\item[normal\_copy] Which is just a plain \ident{for-loop} that copies
    characters from one array to another.
\item[duffs\_device] This is the brain teaser called "Duff's Device", named
    after Tom Duff, the person to blame for this delicious evil.
\item[zeds\_device] A version of "Duff's Device" that just uses a goto so
    you can get a clue about what's happening with the weird \ident{do-while}
    placement in \func{duffs\_device}.
\end{description}

Study these three functions before continuing.  Try to explain what's
going on to yourself before continuing.


\section{What You Should See}

There's no output from this program, it just runs and exits.  You should
run it under valgrind and make sure there are no errors.

\section{Solving The Puzzle}

The first thing to understand is that C is rather loose regarding some
of its syntax.  This is why you can put half of a \ident{do-while} in
one part of a \ident{switch-statement}, then the other half somewhere
else and it will still work.  If you look at my version with the \ident{goto again}
it's actually more clear what's going on, but make sure you understand
how that part works.

The second thing is how the default fallthrough semantics of
\ident{switch-statements} means you can jump to a particular case, and
then it will just keep running until the end of the switch.

The final clue is the \verb|count % 8| and the calculation of \ident{n} at
the top.

Now, to solve how these functions work, do the following:

\begin{enumerate}
\item Print this code out so you can write on some paper.
\item On a piece of paper, write each of the variables in a table as they
    are when they get initialized right before the \ident{switch-statement}.
\item Follow the logic to the switch, then do the jump to the right case.
\item Update the variables, including the \ident{to}, \ident{from}, and the
    arrays they point at.
\item When you get to the \ident{while} part or my \ident{goto} alternative,
    check your variables and then follow the logic either back to the
    top of the \ident{do-while} or to where the \ident{again} label is 
    located.
\item Follow through this manual tracing, updating the variables, until
    you are sure you see how this flows.
\end{enumerate}

\subsection{Why Bother?}

When you've figured out how it actually works, the final question is: Why would
you ever want to do this?  The purpose of this trick is to manually do "loop
unrolling".  Large long loops can be slow, so one way to speed them up is to
find some fixed chunk of the loop, and then just duplicate the code in the loop
out that many times sequentially.  For example, if you know a loop runs a
minimum of 20 times, then you can put the contents of the loop 20 times in the
source code.

Duff's device is basically doing this automatically by chunking up the loop
into 8 iteration chunks.  It's clever and actually works, but these days a good
compiler will do this for you.  You shouldn't need this except in the rare case
where you have \emph{proven} it would improve your speed.

\section{Extra Credit}

\begin{enumerate}
\item Never use this again.
\item Go look at the Wikipedia entry for "Duff's Device" and see if you can
    spot the error.  Compare it to the version I have here and read the article
    carefully to try to understand why the Wikipedia code won't work for you
    but worked for Tom Duff.
\item Create a set of macros that lets you create any length device like this.
    For example, what if you wanted to have 32 case statements and didn't want
    to write out all of them? Can you do a macro that lays down 8 at a time?
\item Change the \func{main} to conduct some speed tests to see which one is
    really the fastest.
\item Read about \func{memcpy}, \func{memmove}, \func{memset}, and also compare
    their speed.
\item Never use this again!
\end{enumerate}

