\chapter{Exercise 2: Make Is Your Python Now}

In \href{http://learnpythonthehardway.org}{Python} you ran programs by just
typing \verb|python| and the code you wanted to run.  The Python interpreter
would just run them, and import any other libraries and things you needed
on the fly as it ran.  C is a different beast completely where you have to
\emph{compile} your source files and manually stitch them together into
a binary that can run on its own.  Doing this manually is a pain, and in
the last exercise you just ran \file{make} to do it.

In this exercise, you're going to get a crash course in GNU make, and you'll
be learning to use it as you learn C.  Make will for the rest of this book,
be your Python.  It will build your code, and run your tests, and set things
up and do all the stuff for you that Python normally does.

The difference is, I'm going to show you smarter Makefile wizardry, where you
don't have to specify every stupid little thing about your C program to get
it to build.  I won't do that in this exercise, but after you've been using
"baby make" for a while, I'll show you "master make".

\section{Using Make}

The first stage of using make is to just use it to build programs
it already knows how to build.  Make has decades of knowledge on building
a wide variety of files from other files.  In the last exercise you
did this already using commands like this:

\begin{Terminal}{Building ex1 with -Wall}
\begin{lstlisting}
$ make ex1
# or this one too
$ CFLAGS="-Wall" make ex1
\end{lstlisting}
\end{Terminal}

In the first command you're telling make, "I want a file named ex1 to 
be created."  Make then does the following:

\begin{enumerate}
\item Does the file \file{ex1} exist already?
\item No. Ok, is there another file that starts with \file{ex1}?
\item Yes, it's called \file{ex1.c}. Do I know how to build \file{.c} files?
\item Yes, I run this command \verb|cc ex1.c   -o ex1| to build them.
\item I shall make you one \file{ex1} by using \file{cc} to build it from \file{ex1.c}.
\end{enumerate}

The second command in the listing above is a way to pass "modifiers" to the
make command.  If you're not familiar with how the Unix shell works, you can
create these "environment variables" which will get picked up by programs you
run.  Sometimes you do this with a command like \verb|export CFLAGS="-Wall"|
depending on the shell you use.  You can however also just put them before the
command you want to run, and that environment variable will be set only while
that command runs.

In this example I did \verb|CFLAGS="-Wall" make ex1| so that it would
add the command line option \verb|-Wall| to the \verb|cc| command that
\ident{make} normally runs.  That command line option tells the compiler
\ident{cc} to report all warnings (which in a sick twist of fate isn't
actually all the warnings possible).

You can actually get pretty far with just that way of using \ident{make},
but let's get into making a \file{Makefile} so you can understand
make a little better.  To start off, create a file with just this
in it:

\begin{code}{A simple Makefile}
\begin{lstlisting}
<< d['code/ex2.1.Makefile|dexy'] >>
\end{lstlisting}
\end{code}

This file is showing you some new stuff with make.  First we set
\ident{CFLAGS} in the file so we never have to set it again, as well
as adding the \verb|-g| flag to get debugging.  Then we have a 
section named \ident{clean} which tells make how to clean up our
little project.

Make sure it's in the same directory as your \file{ex1.c} file, and then
run these commands:

\begin{Terminal}{Running a simple Makefile}
\begin{lstlisting}
$ make clean
$ make ex1
\end{lstlisting}
\end{Terminal}

\section{What You Should See}

If that worked then you should see this:

\begin{Terminal}{Full build with Makefile}
\begin{lstlisting}
<< d['code/ex2.out|dexy'] >>
\end{lstlisting}
\end{Terminal}

Here you can see that I'm running \verb|make clean| which tells
make to run our \ident{clean} target.  Go look at the Makefile
again and you'll see that under this I indent and then I put
the shell commands I want \ident{make} to run for me.  You could
put as many commands as you wanted in there, so it's a great
automation tool.

\begin{aside}{Did You Fix ex1.c?}
If you fixed ex1.c to have \verb|#include <stdio.h>| then your
output will not have the error about puts.  I have the error
here because I didn't fix it.
\end{aside}

Notice also that, even though we don't mention \file{ex1} in the
\file{Makefile}, \ident{make} still knows how to build it \emph{plus}
use our special settings.


\section{How To Break It}

That should be enough to get you started, but first let's break this 
make file in a particular way so you can see what happens.  Take
the line \verb|rm -f ex1| and dedent it (move it all the way left)
so you can see what happens.  Rerun \verb|make clean| and you should
get something like this:

\begin{Terminal}{Bad make run}
\begin{lstlisting}
$ make clean
Makefile:4: *** missing separator.  Stop.
\end{lstlisting}
\end{Terminal}

Always remember to indent, and if you get weird errors like this
then double check you're consistently using tab characters since
some make variants are very picky.

\section{Extra Credit}

\begin{enumerate}
\item Create an \verb|all: ex1| target that will build \file{ex1} with
    just the command \verb|make|.
\item Read \verb|man make| to find out more information on how to run it.
\item Research Makefiles online and see if you can improve this one even more.
\item Find a \file{Makefile} in another C project and try to understand
    what it's doing.
\end{enumerate}

